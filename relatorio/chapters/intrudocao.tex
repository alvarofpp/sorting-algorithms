\chapter{Introdução}
%
% - Algoritmo
%
Um algoritmo é uma sequência finita de passos/instruções, ordenadas de forma lógica, que permitem resolver um determinado problema ou conjunto de problemas de mesmo tipo. Quando tratamos de algoritmo no meio computacional, podemos dividir em 3 partes:
\begin{enumerate}
	\item Entrada de dados;
	\item Processamento;
	\item Saída dos dados resultantes.
\end{enumerate}

%
% - Arranjos
%
No mundo real, lidamos com diversos tipos de dados e um dos modos que possuímos de armazená-los é através de arranjos. Os arranjos são conjuntos/coleções de elementos de tal forma que esses elementos possam ser identificados por um índice ou chave.

Arranjo $A$ de tamanho $n$:
$$
A = 
\begin{bmatrix}
a_{1}, & a_{2}, & a_{3}, & \dots , & a_{n-1}, & a_{n} 
\end{bmatrix}
$$

Identificando elemento do arranjo:
$$ A[1] = a_{1} $$

%
% - Problema da orndeção de arranjos
%
Em determinados casos, precisamos ordenar esses arranjos para facilitar o processamento realizado posteriormente. Esse problema de ordenação é chamado de \textbf{problema da ordenação de um arranjo sequencial}. Como entrada desse problema, temos um arranjo $[ a_{1}, \dots , a_{n} ]$, com $n \in \mathbb{Z}$ e $n > 0$ e a saída é uma permutação $[ a_{\pi1}, \dots , a_{\pi n} ]$ no qual temos a garantia que $a_{\pi 1} \le a_{\pi 2} \le \dots \le a_{\pi n}$.

O presente relatório analisará um total de 7 algoritmos que resolvam o problema citado anteriormente em diferentes casos, sendo eles:
\begin{enumerate}
	\item Insertion sort;
	\item Selection sort;
	\item Bubble sort;
	\item Shell sort;
	\item Quick sort;
	\item Merge sort;
	\item Radix sort (LSD).
\end{enumerate}